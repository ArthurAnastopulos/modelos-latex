% 2020-03-18 - Emerson Ribeiro de Mello
\documentclass[11pt,a4paper]{ifsctech}
\usepackage[utf8]{inputenc}
%-----------------------------%
% Citações padrão ABNT com biblatex
%-----------------------------%
\usepackage[style=abnt,pretty,noslsn]{biblatex}



% Arquivo com a bibliografia
\addbibresource{referencias.bib}


%-----------------------------%
% Para fazer glossário
%-----------------------------%
\GlsXtrLoadResources
[ 
  src = {glossario}, 
  dual-field={dualid},% save label of dual entry in this field
  save-locations=false % para não criar link das páginas onde é referenciado
  % combine-dual-locations=primary % para criar link das páginas onde é referenciado
]

%-----------------------------%
% Capa do relatório
%-----------------------------%
\reportnumber{001/2020}

\title{Modelo de Relatório}

\author{Emerson Ribeiro de Mello}

\date{13 de maio de 2020}

\rodape{{\Huge IFSC}~~~Instituto Federal de Santa Catarina}
%-----------------------------%


\begin{document}
%-----------------------------%

% A semântica da numeração das versões que colocará no relatório deve seguir a orientação descrita no documento [Semantic Versioning 2.0.0](https://semver.org/).

% Dada uma revisão numerada como `MAIOR.MENOR.CORREÇÃO`, incremente esses componentes da seguinte forma:

% 1. `MAIOR` quando as alterações tornarem a nova versão incompatível com a versão anterior;
% 2. `MENOR` quando as alterações adicionarem novos capítulos, seções ou parágrafos e que ainda assim mantém compatibilidade com a versão anterior do documento;
% 3. `CORREÇÃO` quando fizer pequenas correções e ajustes no documento.

%-----------------------------%
% Histórico de revisões do presente documento
%-----------------------------%
\begin{historico}
  \revisao{1.1.0}{13.05.2020}{Adição do ambiente de histórico de versões}
  % \revisao{1.0.1}{04.05.2020}{Correção de erros ortográficos}
  \revisao{1.0.0}{04.05.2020}{Versão inicial}
\end{historico}
%-----------------------------%

% Imprimir sumário
\tableofcontents%
% Imprimir glossário
\printunsrtglossary[type=abbreviations]
\printunsrtglossary
% % Imprimir lista de figuras e de códigos
% \listoffigures%
% \cleardoublepage
% % ---
% % inserir lista de listings no sumário
% % ---
% \pdfbookmark[0]{\lstlistlistingname}{lol}%
% \addcontentsline{toc}{chapter}{\lstlistlistingname}%
% \lstlistoflistings%
% \cleardoublepage%
%-----------------------------%



%-----------------------------%
%	Inicio do documento
%-----------------------------%

\chapter{Introdução}\label{cap:introducao}

Esse documento tem por objetivo apresentar um modelo de relatório técnico para o \gls{IFSC}. Esse modelo faz uso do \texttt{bib2gls} para criar o glossário e lista de acrônimos.  Alguns exemplos podem ser vistos com \gls{xhtml}, \gls{svm} e \gls{matrix}. Em~\cite{rfc2104hmac} está a especificação do \gls{HMAC}.

% -----------------------------
% Referências bibliográficas
% -----------------------------
\printbibliography[heading=bibintoc,title={Referências}]%

\end{document}