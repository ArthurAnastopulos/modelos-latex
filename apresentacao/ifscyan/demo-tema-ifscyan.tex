% !TEX pdfSinglePage
\documentclass{beamer}

\usepackage[utf8]{inputenc}
\usepackage[T1]{fontenc}
\usepackage[english,brazil]{babel}

% usando tema IFSCyan
\usetheme{ifscyan}
\usepackage{beamerthemeifscyan}



% -------------------------------------------------%
% Descomente as linhas abaixo para fazer com que a 
% lista de item se espalhe para ocupar toda a altura
% do slide
% -------------------------------------------------%
% \let\olditem\item
% \renewcommand{\item}{%
% \olditem\vspace{\fill}}     
% -------------------------------------------------%



% -------------------------------------------------%
%              Título 
% -------------------------------------------------%
\title{IFSCyan \\ Um outro tema para o IFSC}
\subtitle{Tema para \LaTeX Beamer}
\author{Prof. Emerson Ribeiro de Mello}
\titlegraphic{\pgfuseimage{ifsclogo}}
\institute{
\href{mello@ifsc.edu.br}{mello@ifsc.edu.br}
}
\date{04/07/2018}
% -------------------------------------------------%


\begin{document}

% -------------------------------------------------%
% Slide com o título
% -------------------------------------------------%
\setbeamertemplate{footline}{} 
\begin{frame}[t]
	\maketitle
\end{frame}
\setbeamertemplate{footline}{
	\pgfuseimage{footer}%
  \usebeamercolor[fg]{framenumber}
  \pgftext[at=\pgfpoint{-35pt}{5pt},center,bottom]{\insertframenumber/\inserttotalframenumber}
}
% -------------------------------------------------%
%                         Inicio do documento
% -------------------------------------------------%

\section{Listas}

\begin{frame}[wide]{Parâmetro wide para aumentar o espaço entre itens}
	\begin{enumerate}
		\item Primeiro item
		\begin{itemize}
			\item \MYhref{http://docente.ifsc.edu.br/mello}{Aqui está um exemplo de link para site}
			\item Segundo item
		\end{itemize}
		\item Segundo item
		\item Terceiro item 
		\begin{itemize}
			\item Primeiro item
			\item Segundo item
		\end{itemize}
	\end{enumerate}
\end{frame}

\begin{frame}[wide]{Listas em colunas}
    \begin{columns}[t, onlytextwidth]
        \column{0.3\textwidth}
            Items:
            \begin{itemize}
                \item Item 1
                \begin{itemize}
                    \item Subitem 1.1
                    \item Subitem 1.2
                \end{itemize}
                \item Item 2
                \item Item 3
            \end{itemize}
        
        \column{0.3\textwidth}
            Enumerations:
            \begin{enumerate}
                \item First
                \item Second
                \begin{enumerate}
                    \item Sub-first
                    \item Sub-second
                \end{enumerate}
                \item Third
            \end{enumerate}
        
        \column{0.4\textwidth}
            Descriptions:
            \begin{description}
                \item[First] Yes.
                \item[Second] No.
            \end{description}
    \end{columns}
\end{frame}


\section{Blocos}


\begin{frame}{Blocos}
	\begin{block}{Esse é um bloco}
		Isso é um teste
	\end{block}
	\begin{block}{}
	Bloco sem título	
	\end{block}
	\begin{alertblock}{Alerta}
		Esse é um alerta
	\end{alertblock}
	\begin{exampleblock}{Bloco para exemplo}
            Exemplo de \textcolor{example}{cor}.
        \end{exampleblock}
\end{frame}

\begin{frame}[wide]{Blocos personalizados}
\begin{atencao}
    Isso é uma mensagem de atenção
\end{atencao}
\begin{informacao}
    Isso é uma mensagem de informação
\end{informacao}
\begin{cuidado}
    Isso é uma mensagem de cuidado
\end{cuidado}

\end{frame}


\begin{frame}[fragile]{Código em C e Java}

\begin{itemize}
		\item Comandos criados para as seguintes linguagens
		\begin{itemize}
			\item \texttt{ansic, java, shell, php, matlab, python, xml, sql}
			\item \texttt{ansicp, javap, shellp, phpp, matlabp, pythonp, xmlp, sqlp}
			\item Letra p no final indica que a fonte será \texttt{scriptsize}
		\end{itemize}
\end{itemize}

% -------------------------------------------------%
% incluindo o código de um arquivo externo
% -------------------------------------------------%
\includecode{ansic}{codigos/ola.c}	


% -------------------------------------------------%
% escrevendo o código diretamente dentro do frame
% -------------------------------------------------%
\javap
\begin{lstlisting}
public static voi main(String args[]){
	System.out.println("Ola mundo");
}
\end{lstlisting}		
\end{frame}

\appendix
\begin{frame}{Referências}
    \nocite{*}
    \bibliography{demo_bibliography}
    \bibliographystyle{plain}
\end{frame}

\begin{frame}{Slide de backup}
    \usebeamercolor[fg]{normal text}
    Slides de backup são úteis para incluir materiais adicionais, necessários somente para ajudar a responder possíveis perguntas da plateia
    \vfill
    O pacote \texttt{appendixnumberbeamer} é usado para não numerar os slides de backup
\end{frame}



\end{document}