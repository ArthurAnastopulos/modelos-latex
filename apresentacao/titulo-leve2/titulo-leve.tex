% !TEX pdfSinglePage
\documentclass{beamer}

\usepackage[utf8]{inputenc}
\usepackage[T1]{fontenc}
\usepackage[english,brazil]{babel}

% usando tema IFSC-Leve
\usepackage{beamerthemeIFSC-LEVE2018}

%\pgfdeclareimage[width=\paperwidth]{footer}{../../latex-estilos/apresentacao/titulo-leve/figs/rodape-2018}
%\pgfdeclareimage[width=.25\paperwidth]{ifsclogo}{../../latex-estilos/apresentacao/titulo-leve/figs/ifsclogo.png}

% Metadados do PDF a ser gerado
\hypersetup{pdfstartview={Fit},pdftitle={\@title},
 	pdfsubject={Engenharia de Telecomunicacoes - IFSC},pdfauthor={\@author}
}

%%%
%%%
% Fazendo com que a lista de item se espalhe para ocupar toda a 
% altura do slide
\let\olditem\item
\renewcommand{\item}{%
\olditem\vspace{\fill}}     
%%%
%%%
%%%



% -------------------------------------------------%
%              Título 
% -------------------------------------------------%
\title{Modelo de apresentação IFSC}
\subtitle{Tema: Título Leve}
\titlegraphic{\pgfuseimage{ifsclogo}}
\author{Prof. Emerson Ribeiro de Mello}
\date{10 de março de 2017}
\institute{\url{mello@ifsc.edu.br}}
% -------------------------------------------------%


\begin{document}

\setbeamertemplate{footline}{} 
\begin{frame}[t]
	\maketitle
\end{frame}
\setbeamertemplate{footline}{
	\pgfuseimage{footer}%
  \usebeamercolor[fg]{framenumber}
  \pgftext[at=\pgfpoint{-35pt}{5pt},center,bottom]{\insertframenumber/\inserttotalframenumber}
}
%--------------------------------------------------------------------------------%
%                         Inicio do documento
%--------------------------------------------------------------------------------%

\section{Listas}

\begin{frame}{Apenas começando}
	\begin{enumerate}
		\item Primeiro item
		\begin{itemize}
			\item Primeiro item
			\item Segundo item
		\end{itemize}
		\item Segundo item
		\item Terceiro item 
		\begin{itemize}
			\item Primeiro item
			\item Segundo item
		\end{itemize}
	\end{enumerate}
\end{frame}

    \begin{frame}{Listas em colunas}
        \begin{columns}[t, onlytextwidth]
            \column{0.33\textwidth}
                Items:
                \begin{itemize}
                    \item Item 1
                    \begin{itemize}
                        \item Subitem 1.1
                        \item Subitem 1.2
                    \end{itemize}
                    \item Item 2
                    \item Item 3
                \end{itemize}
            
            \column{0.33\textwidth}
                Enumerations:
                \begin{enumerate}
                    \item First
                    \item Second
                    \begin{enumerate}
                        \item Sub-first
                        \item Sub-second
                    \end{enumerate}
                    \item Third
                \end{enumerate}
            
            \column{0.33\textwidth}
                Descriptions:
                \begin{description}
                    \item[First] Yes.
                    \item[Second] No.
                \end{description}
        \end{columns}
    \end{frame}


\section{Blocos}


\begin{frame}{Blocos}
	\begin{block}{Esse é um bloco}
		Isso é um teste
	\end{block}
	\begin{block}{}
	Bloco sem título	
	\end{block}
	\begin{alertblock}{Alerta}
		Esse é um alerta
	\end{alertblock}
	\begin{exampleblock}{Bloco para exemplo}
            Exemplo de \textcolor{example}{cor}.
        \end{exampleblock}
\end{frame}


\begin{frame}[fragile]{Código em C e Java}

\begin{itemize}
		\item Comandos criados para as seguintes linguagens
		\begin{itemize}
			\item \texttt{ansic, java, shell, php, matlab, python, xml, sql}
			\item \texttt{ansicp, javap, shellp, phpp, matlabp, pythonp, xmlp, sqlp}
			\item Letra p no final indica que a fonte será \texttt{scriptsize}
		\end{itemize}
\end{itemize}

% incluindo o código de um arquivo externo
\includecode{ansic}{codigos/ola.c}	


% escrevendo o código diretamente dentro do frame
\javap
\begin{lstlisting}
public static voi main(String args[]){
	System.out.println("Ola mundo");
}
\end{lstlisting}		
\end{frame}

\section{Tipos de frames}

\begin{frame}{Simple frame}
Um frame simples
\end{frame}

\begin{frame}[plain]{Plain frame}
   Esse frame está com o tipo \textbf{plain} e sem rodapé
\end{frame}
    
\begin{frame}[t]
Esse frame não tem título e está com o conteúdo alinhado ao topo
\end{frame}
    
\begin{frame}[noframenumbering]{Frame sem rodapé}
 Esse frame não tem rodapé e está citando o trabalho \cite{knuth74}.
\end{frame}
    
    \begin{frame}{Fonte}
        Os pacotes \texttt{inputenc} e \texttt{FiraSans}\footnote{\url{https://fonts.google.com/specimen/Fira+Sans}}\textsuperscript{,}\footnote{\url{http://mozilla.github.io/Fira/}} são usados para definir as fontes
        \vfill
        Esse tema permite os estilo \emph{enfatizado}, \alert{alerta}, \textbf{negrito}, \textcolor{example}{cor de exemplo}, \dots
        \vfill
        \texttt{FiraSans} também provê suporte para símbolos matemáticos:
        \begin{equation*}
            e^{i\pi} + 1 = 0.
        \end{equation*}
    \end{frame}


    \begin{frame}[focus]
        Obrigado!
    \end{frame}
    
    \appendix
    \begin{frame}{Referências}
        \nocite{*}
        \bibliography{demo_bibliography}
        \bibliographystyle{plain}
    \end{frame}
    
    \begin{frame}{Slide de backup}
        \usebeamercolor[fg]{normal text}
        Slides de backup são úteis para incluir materiais adicionais, necessários somente para ajudar a responder possíveis perguntas da plateia
        \vfill
        O pacote \texttt{appendixnumberbeamer} é usado para não numerar os slides de backup
    \end{frame}
\end{document}



\end{document}