% 2017-03-10 - Emerson Ribeiro de Mello - mello@ifsc.edu.br
% \documentclass[handout,xcolor=pdftex,dvipsnames,table]{beamer}
\documentclass{beamer}

% usando tema IFSC-Leve
\usetheme{ifsc-leve}

\usepackage[utf8]{inputenc}
\usepackage[T1]{fontenc}
\usepackage[english,brazil]{babel}

% Metadados do PDF a ser gerado
\hypersetup{pdfstartview={Fit},pdftitle={\@title},
 	pdfsubject={Engenharia de Telecomunicacoes - IFSC},pdfauthor={\@author}
}


% -------------------------------------------------%
%              Título 
% -------------------------------------------------%
\title{Modelo de apresentação IFSC}
\subtitle{Tema: Título Leve}
\author{Prof. Emerson Ribeiro de Mello}
\date{10 de março de 2017}
\institute{Engenharia de Telecomunicações\\
	Instituto Federal de Santa Catarina\\
	campus São José\\
	\url{mello@ifsc.edu.br}
}
% -------------------------------------------------%


\begin{document}

\begin{frame}[t]
	\maketitle
\end{frame}

% Descomente as linhas abaixo se desejar colocar um sumário
%\begin{frame}{Sumário}
%	\tableofcontents
%\end{frame}

%------------------------------------------------------------------------------------%
%                         Inicio do documento
%------------------------------------------------------------------------------------%


\section{Listas}

\begin{frame}{Apenas começando}
	\begin{enumerate}
	% usando o comando espaço para aumentar a distância entre os itens
		\espaco{1.5em}
		\item Primeiro item
		\begin{itemize}
			\item Primeiro item
			\item Segundo item
		\end{itemize}
		\item Segundo item
		\item Terceiro item 
		\begin{itemize}
			\item Primeiro item
			\item Segundo item
		\end{itemize}
	\end{enumerate}
\end{frame}

\subsection{Blocos}


\begin{frame}{Blocos}
	\begin{block}{Esse é um bloco}
		Isso é um teste
	\end{block}
	\begin{block}{}
	Bloco sem título	
	\end{block}
	\begin{alertblock}{Alerta}
		Esse é um alerta
	\end{alertblock}
\end{frame}


\begin{frame}[fragile]{Código em C e Java}

\begin{itemize}
		\item Comandos criados para as seguintes linguagens
		\begin{itemize}
			\item \texttt{ansic, java, shell, php, matlab, python, xml, sql}
			\item \texttt{ansicp, javap, shellp, phpp, matlabp, pythonp, xmlp, sqlp}
			\item Letra p no final indica que a fonte será \texttt{scriptsize}
		\end{itemize}
\end{itemize}

% incluindo o código de um arquivo externo
\includecode{ansic}{codigos/ola.c}	


% escrevendo o código diretamente dentro do frame
\javap
\begin{lstlisting}
public static voi main(String args[]){
	System.out.println("Ola mundo");
}
\end{lstlisting}		
\end{frame}






\end{document}