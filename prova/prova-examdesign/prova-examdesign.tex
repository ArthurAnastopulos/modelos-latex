% 2017-03-18 - Emerson Ribeiro de Mello - mello@ifsc.edu.br
\documentclass[11pt]{examdesign}
\usepackage{estilo-prova-ed}

% ----------------------------------------------------
% Declaração do caminho das imagens
% ----------------------------------------------------
% Logo
\pgfdeclareimage[width=.75cm]{logo}{imagem/logo-mono.png}
% Info
\pgfdeclareimage[width=.6cm]{nota}{imagem/nota.pdf}
% Warning
\pgfdeclareimage[width=.6cm]{atencao}{imagem/atencao.pdf}


% ------------------------------------------------------------------------------------------------------------ %
%                       Códigos fonte em ambiente verbatim ou listing
%
% Ps: se não quiser inserir trechos de códigos, então comente o comando abaixo
% ------------------------------------------------------------------------------------------------------------ %
\IncludeFromFile{codigos/trechos.tex}


% ------------------------------------------------------------------------------------------------------------ %
%                       Para gerar até 26 versões diferentes da mesma prova
%
%                       As questões serão embaralhadas conforme a semente aleatória
% ------------------------------------------------------------------------------------------------------------ %
% Definindo para gerar 3 versões da mesma prova
\NumberOfVersions{3}
% Usar a mesma semente aleatória para sempre gerar sempre a mesma ordem de embaralhamento
% Comente a linha abaixo para gerar sequências diferentes
\setrandomseed{123456789}

% ------------------------------------------------------------------------------------------------------------ %
%                       Folhas com as respostas
% ------------------------------------------------------------------------------------------------------------ %
% Descomente a linha abaixo se não deseja gerar as folhas com as respostas
%\NoKey

% ------------------------------------------------------------------------------------------------------------ %
%                       Dados do cabeçalho da prova
% ------------------------------------------------------------------------------------------------------------ %
% \cabecalho{numero}{data}{Disciplina}{Professor}
\cabecalho{1}{10/03/2017}{STD29006 -- Sistemas Distribuídos}{Emerson Ribeiro de Mello}


% ------------------------------------------------------------------------------------------------------------ %
%                       Início do documento
% ------------------------------------------------------------------------------------------------------------ %
\begin{document}

% Use esse ambiente se desejar colocar uma nota antes do início das questões
\begin{exampreface}

\begin{nota}
Neste documento só tem um pequeno exemplo das funcionalidades da classe \texttt{examdesign}. Veja a documentação da classe para conhecer todas funcionalidades e configurações: \\\url{https://www.ctan.org/tex-archive/macros/latex/contrib/examdesign}
\end{nota}

\end{exampreface}



% ------------------------------------------------------------------------------------------------------------ %
%                       Questões discursivas
% ------------------------------------------------------------------------------------------------------------ %
\begin{shortanswer}[title={Questões discursivas},rearrange=yes,resetcounter=no]
% Se não quiser colocar título é só remover o comando title, como apresentado abaixo
%\begin{shortanswer}[rearrange=yes,resetcounter=no]

\begin{question}[10 pontos]
A transparência é uma das metas para construir um Sistema Distribuído. Quais são os tipos de transparência?
% resposta
	\begin{answer}
	Os tipos são: acesso, localização, desempenho, mobilidade, replicação, concorrência e falhas.
	\end{answer}
\end{question}

\begin{question}[20 pontos]
O agrupamento de máquinas (\textit{cluster}) é um tipo de sistemas de computação distribuídos. Quais são as principais características de um \textit{cluster}?
% resposta
	\begin{answer}
	É formado por computadores semelhantes que geralmente possuem o mesmo sistema operacional e estão conectados por meio de uma rede local.
	\end{answer}
\end{question}

\begin{question}[30 pontos]
O trecho abaixo é de uma implementação de \textit{sockets} na linguagem C. Explique o que acontece na linha 6.
%
% Inserindo trecho de código presente no arquivo codigos/trechos.tex
\InsertChunk{cliente c - sockets}
%
\end{question}

\end{shortanswer}
% ------------------------------------------------------------------------------------------------------------ %

% ------------------------------------------------------------------------------------------------------------ %
%                       Questões verdade/falso
% ------------------------------------------------------------------------------------------------------------ %
\begin{truefalse}[title={Verdade/Falso (10 pontos cada)}, resetcounter=yes]

	\begin{question}
		\answer{Falso} O \textit{cluster} mais potente atualmente está no Brasil.
	\end{question}

	\begin{question}
		\answer{Verdade} Sistemas distribuídos podem possuir arquitetura centralizada, descentralizada ou híbrida.
	\end{question}

\end{truefalse}
% ------------------------------------------------------------------------------------------------------------ %

% ------------------------------------------------------------------------------------------------------------ %
%                       Questões para relacionar
% ------------------------------------------------------------------------------------------------------------ %
\begin{matching}[title={Características das transações}, resetcounter=yes]
Relacione cada característica com sua descrição
  \pair{Atômica}{A transação é indivisível}
  \pair{Consistente}{Toda transação leva o sistema de um estado válido para um outro estado válido}
  \pair{Isolada}{Transações concorrentes não gerem interferência entre si}
  \pair{Durável}{Todas modificações feitas por uma transação são permanentes}

\end{matching}
% ------------------------------------------------------------------------------------------------------------ %




% ------------------------------------------------------------------------------------------------------------ %
%                       Questões de múltipla escolha
% ------------------------------------------------------------------------------------------------------------ %
\begin{multiplechoice}[title={Questões de múltipla escolha}, resetcounter=no]
Marque a opção correta.

\begin{question}
O \textbf{socket} \ldots
    \choice[!]{permite a comunicação entre processos}% resposta
    \choice{já foi muito usado no passado, mas atualmente não é mais usado}
    \choice{permite que a execução de \textit{threads}}
    \choice{está na camada de aplicação}
\end{question}

\end{multiplechoice}
% ------------------------------------------------------------------------------------------------------------ %

% ------------------------------------------------------------------------------------------------------------ %
%                       Mensagem no fim do caderno de questões
% ------------------------------------------------------------------------------------------------------------ %
\begin{examclosing}
\vfill
\begin{center}
\large Boa prova!
\end{center}
\end{examclosing}
% ------------------------------------------------------------------------------------------------------------ %

% ------------------------------------------------------------------------------------------------------------ %
%                       Última folha do exame. Ideal para colocar material de ajuda
% ------------------------------------------------------------------------------------------------------------ %
\begin{endmatter}

\centerline
{\Large Material de apoio para realização da avaliação}
\bigskip
\bigskip

  \begin{center}
  \begin{tabular}{lcl}
    Constant & Symbol & Approximate Value \\ \hline
    Speed of light in vacuum & $c$ & $3.00 \times 10^8$m/s \\
    Permeability of vacuum & $\mu_0$ & $12.6 \times 10^{-7}$H/m \\
    Permittivity of vacuum & $\epsilon_0$ & $8.85 \times 10^{-12}$F/m \\
    Magnetic flux quantum & $\phi_0 = {h \over 2e}$ & $2.07 \times 10^{-15}$Wb \\
    Electron mass & $m_e$ & $9.11 \times 10^{-31}$kg \\
    Proton mass & $m_p$ & $1.673 \times 10^{-27}$kg \\
    Neutron mass & $m_n$ & $1.675 \times 10^{-27}$kg \\
    Proton-electron mass ratio & $m_p \over m_e$ & 1836
  \end{tabular}
  \end{center}
\end{endmatter}
% ------------------------------------------------------------------------------------------------------------ %

 \end{document}
