% 2017-03-18 - Emerson Ribeiro de Mello - mello@ifsc.edu.br
\documentclass[11pt]{article}
\usepackage{estilo-nota-aula}
\usepackage{pdflscape}

% ----------------------------------------------------
% Declaração do caminho das imagens
% ----------------------------------------------------
% Logo
\pgfdeclareimage[width=.75cm]{logo}{imagem/logo-mono.png}
% Info
\pgfdeclareimage[width=.6cm]{nota}{imagem/nota.pdf}
% Warning
\pgfdeclareimage[width=.6cm]{atencao}{imagem/atencao.pdf}


% ------------------------------------------------------------------------------------------------------------ %
%                       Início do documento
% ------------------------------------------------------------------------------------------------------------ %
\begin{document}

\cabecalho{Aula}{9}{23/05/2017}{FIC -- Configuração de Redes de Computadores Linux}{Shell Script}{Emerson Ribeiro de Mello \hfill \url{http://docente.ifsc.edu.br/mello}}

\shell

\section{O interpretador de comandos}

O \textit{shell} consiste em um interpretador de comandos presente em todos os sistemas operacionais variantes dos Unix, que inclui Linux, BSD e MacOS. No linux existem diversos tipos de \textit{shell}, sendo estes: \texttt{csh}, \texttt{bash}, \texttt{ksh} e \texttt{zsh}. 

No interpretador de comandos é possível invocar comandos isolados ou ainda combinar diversos comandos. Por exemplo, o comando \texttt{ls -l} pode ser executado sozinho, porém seria difícil visualizar uma lista grande de arquivos. Assim, o comando \texttt{ls} poderia ser combinado com o comando \texttt{more} o que permite paginar a saída, tornando a leitura mais fácil. Essa combinação de comandos se dá através do uso do \textit{pipe}, representado pelo símbolo $|$. Exemplo: \texttt{ls -l | more}.

\section{Programando em shell}

Como dito, o \textit{shell} é um interpretador de comandos e temos a opção de entrar com uma seqüência de comandos sempre que desejarmos realizar uma tarefa ou podemos colocar tal sequência dentro um arquivo e chamar este arquivo sempre que necessário. E assim temos o \textit{shell script} ilustrado pela Listagem \ref{intro}.

\includecode[shell]{intro}{Meu primeiro shell script}{codigos/intro.sh}

\subsection{Alguns comandos interessantes para shell script}

Abaixo um lista com os principais comandos que iremos utilizar em nossos scripts.

\begin{itemize}
	\item \textbf{echo} -- tem por objetivo imprimir mensagens no dispositivo de saída padrão, no caso o monitor. Abaixo algumas opções:
	\begin{description}
		\item[-e] Ativa a interpretação de caracteres de escape (\texttt{$\backslash$})
		\begin{itemize}
			\item $\backslash$\texttt{n -- nova linha}
			\item $\backslash$\texttt{t -- tab}
			\item $\backslash$\texttt{a alerta (beep)}
		\end{itemize}
		\item[-n] Exibe a mensagem sem pular linha
	\end{description}
	\item \textbf{read} -- Permite que o usuário forneça informações via teclado (é necessário pressionar \texttt{ENTER} para finalizar a leitura). Algumas opções:
	\begin{description}
		\item[-s] não exibe os caracteres que estão sendo fornecidos
		\item[-t seg] aguarda N segundos para que o usuário entre com algum dado
		\item[-n N] Após ler N caracteres o \texttt{read} é encerrado sem que precise pressionar \texttt{ENTER} 
	\end{description}
	\item \textbf{expr} -- para fazer cálculos, porém só faz operações com inteiros. Exemplo de uso: 
	\begin{lstlisting}
	# executando o expr em um terminal
	expr 2 + 2
	# executando o expr em um terminal e guardando o resultado na varial 'soma'
	soma=`expr 2 + 2`
\end{lstlisting}
	\item \textbf{bc} -- trata-se de uma calculadora, ideal para quando necessitamos efetuar cálculos com números reais. Exemplo de uso: 
	\begin{lstlisting}
	# executando o bc em um terminal, combinado com o echo
	echo "scale=2; 1/2" | bc

	# armazendo o resultado da saida do bc na variavel 'resultado'
	resultado=`echo "scale=2; 1/2" | bc`
\end{lstlisting}

\end{itemize}


\subsection{Variáveis}

Nas linguagens de programação as \textbf{variáveis} possuem uma função semelhante com as variáveis da matemática, ou seja, armazenam valores para que possam ser recuperados posteriormente. A Listagem \ref{variaveis} ilustra algumas formas para atribuir e obter valores em variáveis.

\includecode[shell]{variaveis}{Exemplo de definição e uso de variáveis}{codigos/variaveis.sh}

\includecode[shell]{nome}{Usando variáveis em conjunto com o comando read}{codigos/nome.sh}

\newpage

\subsubsection{Variáveis de ambiente}

As \textit{variáveis de ambiente} são aquelas que afetam o comportamento do interpretador de comandos e do \textit{shell script}. É importante frisar que cada \texttt{processo} possui seu ambiente. Um \textit{script} só pode exportar tuas variáveis para os processos filhos. Um \textit{script} invocado através da linha de comando não pode exportar de volta uma variável para o ambiente da linha de comando.

\begin{table}[!htpb]
\begin{center}
\begin{tabular}{|l|p{5cm}|l|p{5cm}|}\hline
\textbf{Variável} & \textbf{Descrição}  & \textbf{Variável} & \textbf{Descrição} \\ \hline \hline 
\$BASH & caminho do binário do bash & \$\$ & número do processo do shell\\ \hline
\$HOME & diretório \textit{home} do usuário & \$HOSTNAME & nome da máquina\\ \hline
\$PATH & caminho para os binários & \$SECONDS & número de segundos desde quando o script começou a ser executado\\ \hline

\end{tabular}
\end{center}
\end{table}


\subsection{Estruturas de decisão}

Antes de apresentar as estruturas de decisão, na tabela \ref{tabelaoperadores} são apresentados os operadores relacionais e lógicos que são de grande importância para tais estruturas.

\begin{table}[!htpb]
 \begin{center}
% use packages: array
\begin{tabular}{|c|c||c|c||c|c|} \hline
\multicolumn{6}{|c|}{\textbf{Operadores lógicos e relacionais}} \\ \hline
\multicolumn{2}{|c||}{\textbf{Numéricos}} & \multicolumn{2}{|c||}{\textbf{Cadeia de caracteres}} & \multicolumn{2}{|c|}{\textbf{Operadores lógicos}}\\ \hline \hline
\texttt{-eq} & igual     & \texttt{=} & igual & \texttt{-a} & E lógico (AND) \\ \hline
\texttt{-ne} & diferente & \texttt{!=} & diferente & \&\& & E lógico (AND)\\ \hline
\texttt{-ge} & maior ou igual & \texttt{-n} & não é nula & \texttt{-o} & OU lógico (OR)\\ \hline
\texttt{-le} & menor ou igual & \texttt{-z} & é nula & $\|$ & OU lógico (OR)\\ \hline
\texttt{-gt} & maior &  & & \texttt{!} & negação \\ \hline
\texttt{-lt} & menor &  &  & &\\ \hline
 \end{tabular}
 \end{center}
\caption{Operadores relacionais e lógicos}
\label{tabelaoperadores}
\end{table}


\subsubsection{Se\ldots então\ldots senão}

\includecode[shell]{se}{Estrutura de decisão SE}{codigos/se.sh}

\clearpage

\includecode[shell]{se2}{Usando operador lógico E}{codigos/se2.sh}

\subsubsection{Escolha\ldots caso\ldots}

\includecode[shell]{caso}{Estrutura de decisão ESCOLHA}{codigos/caso.sh}

\newpage

\subsection{Estruturas de repetição}

\subsubsection{Enquanto}

\includecode[shell]{enquanto}{Estrutura de repetição ENQUANTO}{codigos/enquanto.sh}


\subsubsection{Para}

\includecode[shell]{para}{Estrutura de repetição PARA}{codigos/para.sh}


\section{Exercícios}

\begin{enumerate}
	\item Desenvolva um algoritmo	que leia dois números inteiros e exiba a soma destes números.
	\item Desenvolva um algoritmo que solicite ao usuário seu nome e exiba uma mensagem de boas vindas utilizando este nome.
	\item Desenvolva um algoritmo que leia um número inteiro e determine se este é par ou ímpar.
	\item Desenvolva um algoritmo que leia dois números inteiros e exiba qual deles é o maior 
	\item Desenvolva um algoritmo que leia um número inteiro positivo e imprima a sequência de 0 até este número. 		\item Desenvolva um algoritmo que simule a autenticação de usuários. O usuário deve fornecer uma senha e se esta senha for igual a palavra \textbf{secreta} deverá exibir a mensagem ``Acesso autorizado'', caso contrário deverá exibir ``Acesso negado''. O algoritmo deverá solicitar a senha ao usuário até que este forneça a senha correta ou até que o número de tentativas permitidas seja alcançado. No caso, o número máximo de tentativas é $3$.
\end{enumerate}


 \end{document}